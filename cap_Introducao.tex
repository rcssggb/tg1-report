
\chapter{Introdução}

\label{CapIntro}

% Resumo opcional. Comentar se não usar.
% \resumodocapitulo{Resumo opcional}


\section{RoboCup Soccer Simulation 2D}
\par A ideia de robôs jogando futebol foi proposta pela primeira vez em 1992 por Alan Mackworth\cite{}. 
Desde então a comunidade científica tem criado iniciativas buscando por soluções que torne isso realidade.
Uma delas é a \textit{Robot World Cup Initiative}, abreviada como \textit{RoboCup}, que teve sua primeira edição em 1997 com mais de 40 equipes distribuídas entre as diversas categorias do evento.
Atualmente, a RoboCup conta com mais de 10 categorias, entre elas a \textit{RoboCup Soccer Simulation 2D}, abreviada RCSS, objeto de estudo deste projeto. 
\par A categoria apresenta, também, grande relevância no cenário brasileiro. 
Desde 2005, a RCSS está presente na maior competição de robótica da América Latina, a \textit{Latin American Robotics Competition}, LARC.
\par Nessa categoria, duas equipes de 11 jogadores autônomos e independentes jogam futebol em um ambiente virtual bidimensional. 
Um servidor é responsável por esse ambiente e possui informação absoluta sobre o estado do jogo e suas regras.
Os jogadores, por sua vez, recebem dele informação incompleta e ruidosa de seus sensores virtuais, podendo executar comandos a fim de atuar sobre o estado do jogo.

\section{Abordagens utilizadas na categoria}
\par Uma pesquisa sobre as abordagens para o desenvolvimento das estratégias dos times participantes da RCSS revelou o uso recorrente de métodos de inteligência computacional.
\par A equipe chinesa \textit{WrightEagle}, campeã da RoboCup diversas vezes, utiliza Processos de Decisão de Markov ou MDPs para modelar a partida\cite{}.
\par A equipe japonesa \textit{HELIOS}, campeã de 2018 da categoria na RoboCup, divide seus jogadores em categorias "chutadores" e "não-chutadores". 
Os chutadores são responsáveis por realizar o planejamento de sequência de ações, utilizando métodos de ação-valor.
Os não-chutadores, por sua vez, não tem conhecimento do planejamento feito pelos chutadores, e devem obter o máximo de informações relevantes para tentar gerar a mesma sequência de ações que jogador chutador\cite{}.
\par A equipe brasileira \textit{ITAndroids}, atual campeã da LARC, utiliza a abordagem de sequência de ações, similar à \textit{HELIOS}, explorando uma árvore de ações criada dinamicamente de forma a maximizar o valor de cada ação. Além disso, utilizam Otimização por Enxame de Partículas para adequar os parâmetros que calculam o valor da ação.