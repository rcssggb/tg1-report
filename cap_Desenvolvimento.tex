\chapter{Desenvolvimento \label{chap:Desenvolvimento}}

% Resumo opcional. Comentar se não usar.
% \resumodocapitulo{Resumo opcional.}


\section{Biblioteca}
\par O servidor da partida apresenta, como já mencionado, um protocolo de comunicação e sintaxe de mensagens específica. Uma biblioteca de interfaceamento é proposta com o objetivo de abstrair os detalhes de comunicação de construção de mensagens e facilitar, assim, o desenvolvimento dos jogadores. Esta abordagem já é comum na categoria e existem soluções de código aberto como a \textit{librcsc}, utilizada por várias equipes, usualmente atreladas ao agente base \textit{agent2d}, desenvolvidas pela equipe \textit{HELIOS}.
\par Uma biblioteca própria desenvolvida em linguagem Go é proposta como forma de modernização da base utilizada pelas equipes.

\section{Agente Único}
\par Inicialmente, deseja-se realizar o treinamento de um agente único que, com as informações de seus sensores, consiga com sucesso levar a bola ao gol. Essa proposta tem como objetivo construir a base para a realização de um treinamento de um time de múltiplos agentes, em um estágio posterior.
\par O desenvolvimento de um agente único, inicialmente, permite adquirir o entendimento necessário para a definição do vetor de estados e técnicas de treinamento. Além disso, busca-se uma maior agilidade na substituição e teste na estrutura do vetor de estados e no algoritmo de treinamento.

\subsection{Vetor de Estados}

\section{Múltiplos Agentes}

\subsection{Vetor de Estados}


