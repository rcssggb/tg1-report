\chapter{Desenvolvimento \label{chap:Desenvolvimento}}

% Resumo opcional. Comentar se não usar.
% \resumodocapitulo{Resumo opcional.}


\section{Biblioteca}
\par O servidor da partida apresenta, como já mencionado, um protocolo de comunicação e sintaxe de mensagens específica. Uma biblioteca de interfaceamento é proposta com o obejtivo de abstrair os detalhes de comunicação de construção de mensagens e facilitar, assim, o desenvolvimento dos jogadores. Esta abordagem já é comum na categoria e existem soluções de código aberto como a \textit{librcsc}, utilizada por várias equipes, usualmente atreladas ao agente base \textit{agent2d}.

\section{Agente Único}

\subsection{Vetor de Estados}

\section{Múltiplos Agentes}

\subsection{Vetor de Estados}


