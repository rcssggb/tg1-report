
\chapter{Conclusões}

\label{CapConclusoes}


\section{Reimplementação da Interface com o Servidor}
\par Ao pesquisar sobre a comunidade e equipes que participam das edições nacionais da competição, notou-se que a biblioteca de interfaceamento com o servidor, \textit{librcsc}, e o time base, \textit{agent2d} - ambos desenvolvidos no Japão por acadêmicos relacionados à equipe \textit{HELIOS} - são amplamente utilizados. Entretanto, a documentação da biblioteca é escassa e há dificuldade de utilização dela, evidenciado por conversas com os participantes.
\par Esse cenário demonstra a necessidade de modernização da base de código utilizada pelas equipes.
É proposto, então, a reimplementação da interface com o servidor da partida utilizando a linguagem Go.

\section{Treinamento de uma Equipe para Participação em Competições}
 \par Este projeto propõe o treinamento de um time capaz de competir contra as principais equipes nacionais e internacionais da categoria. 
 Serão estudados, avaliados e implementados diversos métodos de inteligência computacional para o treinamento do time a fim de chegar a um resultado satisfatório e coerente com os objetivos propostos. 
 