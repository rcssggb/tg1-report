\chapter{Fundamentação Teórica \label{chap:FundamentacaoMatematica}}

% Resumo opcional. Comentar se não usar.
% \resumodocapitulo{Resumo opcional.}


\section{Processos de Decisão de Markov}

O problema abordado neste trabalho pode ser descrito como um Processo de Decisão de Markov (MDP).
MDP é uma forma clássica de representação matemática de processos de decisão sequenciais.
Nessa representação, cada ação tomada por um agente que interage com o ambiente transforma o estado do processo e determina a recompensa que o agente recebe imediatamente.
Esse estado também deve ser suficiente para conter toda a informação relevante para a dinâmica futura do processo.

% TODO inserir imagem clássica environment -> state -> agent -> action -> environment -> reward -> agent

Assim, dado um espaço de estados $\mathcal{S}$, um espaço de ações $\mathcal{A}$ e um espaço de recompensas $\mathcal{R}$, para cada par $(S, A)$ com $S \in \mathcal{S}$ sendo o estado atual do processo e $A \in \mathcal{A}$ a ação tomada pelo agente existe uma determinada probabilidade de atingir o estado $S' \in \mathcal{S}$ e receber a recompensa imediata $R \in \mathcal{R}$ \cite{sutton2018reinforcement}.

Essa abordagem é bastante flexível e torna possível a modelagem da dinâmica do futebol virtual de robôs de diversas maneiras de modo que cada agente possa construir um estado percebido a partir de seus sensores e tomar decisões acerca de qual ação tomar diante desse estado a fim de maximizar a recompensa recebida.

\subsection{MDP Episódico e Contínuo}

Um MDP pode ser caracterizado quanto à presença de um estado terminal. Caso o MDP tenha um ou mais estados que determinem o fim do processo, ele é dito episódico. A simulação de futebol de robôs tratada neste trabalho é um exemplo de MDP episódico, uma vez que o MDP termina ao se encerrar o tempo de jogo.

Em contrapartida, há MDPs onde não está bem definido nenhum estado terminal. Nesses casos, o MDP pode continuar indefinidamente até que uma ação externa ao MDP determine a sua parada. Um exemplo disso é um MDP que controle um robô numa linha de produção. Caso o sistema de automação supervisor desse robô não determine sua parada (por falta de insumos, por exemplo), o MDP pode seguir operando indefinidamente.

Nesta fundamentação, será tratado com mais atenção o caso episódico uma vez que é o caso que interessa para aplicação na simulação de futebol de robôs.

\subsection{Recompensa e Retorno}

Como definido acima, para cada ação tomada em um MDP é atribuída uma recompensa $R \in \mathcal{R}$. Essa recompensa é sempre referente ao instante de tempo anterior, ou seja, não depende de qualquer outro fator que não o par $(S_t, A_t)$ executados no instante $t$ e a função de probabilidade associada pelo MDP a esse par. Por isso, é comum utilizar a notação $R_{t+1}$ para se referir à recompensa obtida após tomar a ação $A_t$ no instante de tempo $t$.

Porém em muitos casos é esperado de um agente que ele tome decisões que maximizem a recompensa total ao fim de um episódio, ou seja, é esperado que se escolha $A_t$ a fim de maximizar não apenas $R_{t+1}$ mas sim o retorno $G_{t+1} = R_{t+1} + R_{t+2} + \dotsc + R_{terminal}$.

% TODO pra encher linguiça a gente pode definir retorno descontado por lambda

\subsection{Políticas}

É dado o nome de política para qualquer função $\pi(S) \to \mathcal{A}$ que leve de um estado qualquer do MDP para uma ação a ser tomada. Para cada política $\pi$, existe uma função $q_\pi(S, A)$ que, para cada par de estado e ação, define a esperança de retorno caso o agente continue seguindo a política $\pi$ no restante do episódio.

É possível comparar duas políticas $\pi$ e $\pi'$ a respeito de suas funções $q$. A política $\pi$ é considerada melhor ou igual a $\pi$, ou $\pi \ge \pi'$, caso $q_\pi(S, A) \ge q_{\pi'}(S, A)$ para todo par $(S, A)$.

Sempre há ao menos uma política melhor ou igual a todas as outras, denominada política ótima. Qualquer política que cumpra esse requisito é denominada $\pi_*$ e, caso haja mais de uma, todas devem possuir a mesma função $q$ denominada $q_*$ \cite{sutton2018reinforcement}. % chap 3.6

Uma política que toma sempre o caminho de maior retorno é denominada gulosa, e uma política que toma o caminho de maior retorno mas escolhe uma ação aleatoriamente com probabilidade parametrizada $\epsilon$ é denominada $\epsilon$-gulosa.

\section{Aprendizagem por Reforço}

Dada uma modelagem do problema como um MDP, resta obter uma maneira de estimar as probabilidades que determinam a dinâmica desse MDP, e com isso determinar um critério de decisão - denominado política - capaz de maximizar a recompensa a longo prazo recebida pelo agente.

O conjunto de técnicas que resolvem esse tipo de problema é chamado de Aprendizagem por Reforço.
No campo da aprendizagem de máquina, ela se difere da Aprendizagem Supervisionada por não haver um conjunto de pares $(s, a)$ dados como corretos.
Nesse tipo de aprendizagem, o objetivo é extrapolar uma solução genérica a partir de exemplos de um conjunto de treinamento dado como correto, o que não é prático em problemas em que não se tem exemplos de comportamentos corretos e que representem bem o conjunto total de situações possíveis.
Ela também se diferencia da Aprendizagem Não-Supervisionada, que tradicionalmente visa encontrar estrutura em conjuntos de dados não classificados, enquanto a Aprendizagem por Reforço visa maximizar um sinal de recompensa \cite{sutton2018reinforcement}.

Desse modo, as técnicas de Aprendizagem por Reforço serão aplicadas a fim de buscar políticas capazes de maximizar o desempenho dos jogadores virtuais, ou seja, obter políticas que tornem os agentes capazes de fazer gols e evitar que os jogadores do time adversário façam gols.

\subsection{Aprendizagem On-policy e Off-policy}

Entre as técnicas de aprendizagem por reforço existe uma divisão entre a aprendizagem on-policy e a aprendizagem off-policy, referentes à relação entre a política executada durante o aprendizado e a política sobre a qual se quer aprender.

Nos algoritmos de aprendizagem on-policy, o agente aprende a respeito da política $\pi$ enquanto navega o MDP de acordo com a própria política $\pi$.

Já nos algoritmos de aprendizagem off-policy, o agente aprende a respeito da política alvo $\pi$ enquanto navega o MDP de acordo com a política $b$, ou seja, ele estima a função $q_{\pi}$ enquanto segue a política $b$.

Os métodos off-policy costumam introduzir variância no processo, tornando o aprendizado ruidoso e em alguns casos a garantia de convergência é provada apenas para o caso on-policy. % TODO: carece de fonte

Além disso, é possível observar que a aprendizagem on-policy é apenas um caso particular da aprendizagem off-policy em que $b = \pi$.

\subsection{Métodos Tabulares e de Solução Aproximada}

% TODO

\subsection{Q-Learning}

Um dos algoritmos mais populares no campo da aprendizagem por reforço é o Q-Learning. Trata-se de um método off-policy que aproxima diretamente a função $q_*$ independente da política que estiver sendo adotada pelo agente durante o treinamento.

O algoritmo é também muito simples. Dada uma representação tabular $Q: (\mathcal{S},\mathcal{A}) \to \mathbb{R}$ da função $q_*$, para cada instante de tempo $t$ é realizada a seguinte atualização a fim de aproximar $Q$ de $q_*$:

\begin{equation}
Q(S_t, A_t) \leftarrow Q(S_t, A_t) + \alpha[R_{t+1} + \max_{a} Q(S_{t+1}, a) - Q(S_t, A_t)]
\end{equation}

Após iterações suficientes, espera-se que $Q$ convirja para $q_*$. Em alguns casos a convergência é provada matematicamente.

Uma vez estimada a função $q_*$, é simples obter a política ótima. Basta escolher a ação que maximiza $q_*$ no estado atual, ou seja:

\begin{equation}
A_{t+1} = \max_{a} q_*(S_t, a)
\end{equation}

É comum, mas não obrigatório, que a política $b$ seguida durante o aprendizado seja $epsilon$-gulosa em relação à aproximação Q.

\subsection{Q-Learning Duplo}

Apesar de popular o Q-Learning possui um problema de viés de maximização. Uma vez que a aproximação $Q$ é imprecisa no início do treinamento, é possível que o retorno esperado estimado seja enviesado para um valor maior do que o real.

Como solução para esse problema, é utilizada a abordagem do Q-Learning Duplo. Nela são utilizadas duas aproximações, $Q_1$ e $Q_2$, e a atualização de $Q$ é dada da seguinte forma:

\begin{equation}
\label{eq:doubleq}
Q_1(S_t, A_t) \leftarrow Q_1(S_t, A_t) + \alpha[R_{t+1} + Q_2(S_{t+1}, \text{arg}\max_a Q_1(S_{t+1}, a)) - Q_1(S_t, A_t)]
\end{equation}

Em metade das iterações (através de um sorteio, por exemplo), as aproximações $Q_1$ e $Q_2$ são trocadas. Com isso é anulado o viés de maximização gerado pelo uso de $\max_a Q$ como estimativa de retorno para os estados seguintes.

A vantagem desse método é que apesar de dobrar os requisitos de memória do algoritmo, afinal será preciso armazenar os dados referentes a duas aproximaçoes, ele não aumenta o custo computacional por iteração.
